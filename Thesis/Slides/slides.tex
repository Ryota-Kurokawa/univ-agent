\documentclass[aspectratio=169,12pt]{beamer}
\usetheme{Madrid}
\usecolortheme{default}

% LuaLaTeX用日本語設定
\usepackage{luatexja}
\usepackage[haranoaji]{luatexja-preset}

\usepackage{amsmath,amssymb}
\usepackage{booktabs}

\title{中間管理職の負荷問題に関する組織構造研究}
\subtitle{エージェントベースシミュレーションによる定量的分析}
\author{黒川 良太}
\institute{古谷研究室}
\date{2026年1月}

\begin{document}

% スライド1: タイトル
\begin{frame}
\titlepage
\end{frame}

% スライド2: 研究背景・問題意識
\begin{frame}{研究背景・問題意識}
\begin{itemize}
\item 現代の知識集約型企業において、中間管理職の業務負荷が過度に高まっている
\item 中間管理職は上位層からの戦略的指示と現場からの報告・相談を同時に処理する「サンドイッチ状態」
\item タスク分割、調整、報告受信といった\textbf{不可視の業務}が集中
\end{itemize}

\vspace{0.5cm}
\begin{block}{従来の捉え方}
個人の能力不足として扱われることが多い
\end{block}

\begin{alertblock}{本研究の視点}
組織構造とタスクフローに起因する\textbf{構造的問題}として捉える
\end{alertblock}
\end{frame}

% スライド3: 研究目的
\begin{frame}{研究目的}
\begin{enumerate}
\item \textbf{中間管理職への負荷集中の定量的確認}
  \begin{itemize}
  \item SimLoad(シミュレーション負荷)を用いた定量化
  \end{itemize}

\item \textbf{管理スパン変更が負荷分布に与える影響の解明}
  \begin{itemize}
  \item 管理スパン(Span of Control)を段階的に変更
  \end{itemize}

\item \textbf{負荷軽減のための組織構造最適化手法の提案}
  \begin{itemize}
  \item 2つの異なるアプローチを比較検証
  \end{itemize}
\end{enumerate}
\end{frame}

% スライド4: 提案手法(概要)
\begin{frame}{提案手法:概要}
\begin{block}{エージェントベースシミュレーション}
組織をグラフとしてモデル化し、タスクの流れを再現
\end{block}

\begin{columns}
\begin{column}{0.5\textwidth}
\textbf{組織モデル}
\begin{itemize}
\item 5層階層組織
\item 総ノード数:約25,000
\item CXO → Director → SeniorManager → Manager → Player
\end{itemize}
\end{column}

\begin{column}{0.5\textwidth}
\textbf{TaskAgent}
\begin{itemize}
\item タスクを表現
\item 組織内を移動・処理
\item 負荷を定量化
\end{itemize}
\end{column}
\end{columns}

\vspace{0.5cm}
\begin{itemize}
\item パラメータ:IT企業3名へのインタビュー調査に基づく
\item TIME\_STEPS=2000、seed=42で再現性確保
\end{itemize}
\end{frame}

% スライド5: 提案手法(詳細)
\begin{frame}{提案手法:SimLoad の計算}
\begin{block}{SimLoad(シミュレーション負荷)の定義}
各ノードがタスクを処理する際の負荷を以下のように定量化:
\end{block}

\begin{align*}
\text{タスク受信} &: +1.0 \times w \\
\text{タスク送信} &: +0.3 \times w \\
\text{報告受信} &: +0.1 \times w \\
\text{分割コスト} &: +0.2 \times w
\end{align*}

\vspace{0.3cm}
ここで、$w$ は TaskWeight(難易度、関係者数、調整コスト、曖昧さの総和)

\vspace{0.3cm}
\begin{itemize}
\item 各係数は本研究で独自に設定
\item タスク受信を基準負荷(1.0)として、送信・報告の調整コストを相対評価
\end{itemize}
\end{frame}

% スライド6: 実験設計
\begin{frame}{実験設計:2つのアプローチ}
\begin{columns}
\begin{column}{0.48\textwidth}
\begin{block}{実験A:効率化}
\textbf{SM人数固定}で管理スパン削減
\end{block}
\begin{itemize}
\item SM→Mgr: [10,15] → [4,6]
\item Mgr→Player: [5,10] → [18,22]
\item SM人数:238名(固定)
\item 狙い:1人あたりの処理効率向上
\end{itemize}
\end{column}

\begin{column}{0.48\textwidth}
\begin{block}{実験B:分散化}
\textbf{SM人数増加}で総負荷分散
\end{block}
\begin{itemize}
\item Dir→SM: [10,15] → [18,23]
\item SM→Mgr: [10,15] → [4,6]
\item SM人数:238名 → 398名
\item 狙い:総負荷の分散
\end{itemize}
\end{column}
\end{columns}

\vspace{0.5cm}
\begin{itemize}
\item 両実験とも4段階(Step1~4)で段階的に変更
\item InterviewBased シナリオをベースライン
\item 総ノード数を約25,000に維持(組織規模の影響を排除)
\end{itemize}
\end{frame}

% スライド7: 結果(表)
\begin{frame}{主要な結果:数値比較}
\begin{table}
\centering
\begin{tabular}{lccc}
\toprule
& Baseline & 実験A (Step4) & 実験B (Step4) \\
\midrule
SM人数 & 238名 & 238名 & 398名 \\
SM平均SimLoad & 94,420 & 81,562 & 74,369 \\
\textbf{変化率} & --- & \textbf{-14\%} & \textbf{-21\%} \\
\midrule
SM正規化SimLoad & 2.67 & 2.19 & 3.21 \\
\textbf{変化率} & --- & \textbf{-18\%} & \textbf{+20\%} \\
\midrule
Mgr平均タスク数 & 56,625 & 139,376 & 83,313 \\
\textbf{変化率} & --- & \textbf{+146\%} & \textbf{+47\%} \\
\bottomrule
\end{tabular}
\end{table}

\vspace{0.3cm}
\begin{itemize}
\item 両実験とも SM の絶対負荷は減少
\item ただし、正規化負荷(効率)と総負荷分散のトレードオフが存在
\end{itemize}
\end{frame}

% スライド8: 結果(まとめ)
\begin{frame}{主要な結果:4つの発見}
\begin{enumerate}
\item \textbf{管理スパン削減は有効}
  \begin{itemize}
  \item 両実験とも SM→Mgr スパン削減により SM の絶対負荷は減少
  \item 実験A: 14\%減、実験B: 21\%減
  \end{itemize}

\item \textbf{実験A:処理効率向上}
  \begin{itemize}
  \item SM の1タスクあたり負荷が18\%減少
  \item ただし Mgr のタスク数が146\%増加(負荷遷移)
  \end{itemize}

\item \textbf{実験B:総負荷分散}
  \begin{itemize}
  \item SM 人数が67\%増加し、絶対負荷は21\%減少
  \item ただし1タスクあたり負荷は20\%増加(効率低下)
  \end{itemize}

\item \textbf{負荷均衡点の存在}
  \begin{itemize}
  \item Step4で SM と Mgr の最高 SimLoad がほぼ均衡
  \item 実験A: 90,780 vs 84,767、実験B: 82,670 vs 82,174
  \end{itemize}
\end{enumerate}
\end{frame}

% スライド9: 考察
\begin{frame}{考察:トレードオフの存在}
\begin{block}{本質的なトレードオフ}
管理スパン最適化には、相反する2つの効果が存在
\end{block}

\begin{columns}
\begin{column}{0.48\textwidth}
\begin{alertblock}{実験A}
「Mgrへの負荷遷移」
\end{alertblock}
\begin{itemize}
\item SM の効率は向上
\item Mgr の負荷が大幅増加
\item \textbf{適用場面}:Mgr層に余裕がある組織
\end{itemize}
\end{column}

\begin{column}{0.48\textwidth}
\begin{alertblock}{実験B}
「1人あたり効率の低下」
\end{alertblock}
\begin{itemize}
\item 総負荷は分散
\item 1人あたりの効率低下
\item \textbf{適用場面}:負荷分散を優先する組織
\end{itemize}
\end{column}
\end{columns}

\vspace{0.5cm}
\begin{itemize}
\item 組織の総業務量は一定
\item 特定の役職層の負荷だけを削減することは困難
\end{itemize}
\end{frame}

% スライド10: 結論
\begin{frame}{結論}
\begin{block}{本研究の主要な貢献}
中間管理職への負荷集中は組織構造に起因する\textbf{構造的問題}であることを定量的に示した
\end{block}

\begin{enumerate}
\item \textbf{負荷集中の定量化}
  \begin{itemize}
  \item SeniorManager 層の平均 SimLoad が最も高い(94,420)
  \end{itemize}

\item \textbf{2つの最適化アプローチの提示}
  \begin{itemize}
  \item 効率化(実験A)と分散化(実験B)の効果とトレードオフを明確化
  \end{itemize}

\item \textbf{適切な管理スパンの確認}
  \begin{itemize}
  \item 5-6人程度が効率的であることを確認
  \end{itemize}

\item \textbf{定量的フレームワークの提供}
  \begin{itemize}
  \item 組織設計者が科学的根拠に基づいた意思決定を行うための基盤
  \end{itemize}
\end{enumerate}
\end{frame}

% スライド11: 今後の課題
\begin{frame}{限界と今後の課題}
\begin{block}{本研究の限界}
\begin{itemize}
\item 5層階層のツリー構造に限定(マトリクス型組織などは未対応)
\item パラメータは限定的なインタビューに基づく設定
\item 組織の動的変化(人員増減、再編成など)は未モデル化
\end{itemize}
\end{block}

\begin{block}{今後の研究方向性}
\begin{enumerate}
\item 実際の企業データを用いた検証
  \begin{itemize}
  \item 負荷均衡点が実組織でも観測されるかを確認
  \end{itemize}
\item 実験A・Bの組み合わせによる最適解探索
  \begin{itemize}
  \item 多目的最適化手法の適用
  \end{itemize}
\item より多様な組織構造への拡張
  \begin{itemize}
  \item マトリクス型、ネットワーク型組織
  \end{itemize}
\item AI エージェントによる業務補助の効果検証
\end{enumerate}
\end{block}
\end{frame}

% スライド12: Thank you
\begin{frame}
\centering
\Huge Thank you for your attention!

\vspace{1cm}
\Large
ご清聴ありがとうございました

\vspace{1cm}
\normalsize
質問・コメントをお待ちしております
\end{frame}

\end{document}
