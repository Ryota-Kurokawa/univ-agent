\chapter{背景と問題設定}

\section{現代の働き方と中間管理職を取り巻く環境}

2020年以降、COVID-19パンデミックを契機として、働き方は急激に変化した。リモートワークやハイブリッドワークが急速に普及し、オフィスに全員が集まることを前提とした従来のマネジメント手法は通用しなくなった。この変化は一時的なものではなく、多くの企業で恒久的な働き方として定着しつつある。従業員は柔軟な働き方を求め、企業は優秀な人材を確保するためにリモートワークを許容せざるを得ない状況にある。

働き方の変化に伴い、コミュニケーション手段も多様化した。Slack、Microsoft Teams、Zoom、メール、電話、対面会議など、複数のチャネルが並行して使用されるようになった。この多様化は利便性をもたらす一方で、新たな問題を生んでいる。中間管理職は、これらすべてのチャネルを監視し、適切なタイミングで適切な手段を選択して対応することを求められる。通知の洪水の中で、重要な情報を見逃さないよう常に注意を払う必要があり、これが大きな認知的負荷となっている。

日本では働き方改革関連法の施行により、労働時間の上限規制が厳格化された。長時間労働の是正は労働者保護の観点から重要な施策である。しかし現実には、業務量が減少しないまま労働時間だけが制限されるケースが多い。その結果、限られた時間内でより多くの業務をこなすことが求められ、時間あたりの負荷密度は増加している。特に中間管理職は、部下の労働時間を管理しながら自身も時間制限を守る必要があり、この矛盾の中で板挟みになりやすい。

近年、意思決定の迅速化やコスト削減を目的として、組織のフラット化が進んできた。階層を減らすことで情報伝達の遅延を防ぎ、現場の声を経営に届きやすくする狙いがある。しかし、フラット化の結果として、中間管理職一人あたりが管理する部下の数(管理スパン)は拡大する傾向にある。また、管理職でありながら自身もプレイヤーとして実務をこなす「プレイングマネージャー」が増加している。マネジメント業務と実務の両立は、時間的にも認知的にも大きな負担となる。

\section{中間管理職の「負荷」とは何か}

従来、業務負荷は主に「タスクの量」や「労働時間」で測られてきた。しかし、現代の知識労働において、この捉え方には限界がある。同じ1時間でも、集中して一つの作業に取り組む1時間と、細切れの会議や割り込み対応に追われる1時間では、消耗度が全く異なる。また、単純作業を10件こなすのと、複雑な調整を1件行うのでは、後者の方がはるかに精神的な負担が大きい場合がある。

認知的負荷とは、情報処理や意思決定に必要な精神的リソースの消費を指す。中間管理職は日常的にコンテキストスイッチ、情報の統合、不確実性への対処といった認知的負荷にさらされている。異なるプロジェクトや案件の間を頻繁に行き来することで思考の切り替えコストが発生し、一度中断した作業に戻る際には状況を思い出し思考を再構築する時間が必要となる。また、複数の部下からの報告、上司からの指示、他部署からの依頼など、様々なソースからの情報を統合し全体像を把握する必要がある。さらに、十分な情報がない中で判断を下さなければならない場面が多く、その不確実性自体が精神的な負担となる。

調整負荷とは、複数の関係者間の利害や意見を調整するために必要なリソースの消費を指す。中間管理職は組織の中間に位置するため、調整業務が集中する。上下間の橋渡しとして、経営層からの戦略的・抽象的な指示を現場が実行可能な具体的なタスクに翻訳し、逆に現場の状況や課題を経営層に理解できる形で報告する。部門間の調整では、自部門と他部門の間で発生する依存関係や利害対立を調整し、リソースの奪い合いや優先順位の衝突を解決する役割を担う。部下間の調整では、部下同士の業務分担、スキルの違い、人間関係の問題などを調整し、チームとして機能するようにまとめる。

中間管理職の業務の多くは、成果物として目に見える形で残らない不可視の負荷である。根回しや事前調整として、正式な会議の前に関係者と個別に話を通し合意形成を円滑にするための非公式なコミュニケーションが必要となる。また、なぜこの方針なのか、なぜこの優先順位なのかを上司にも部下にも繰り返し説明する作業や、部下のモチベーション管理、不満への対応、チームの雰囲気づくりなどの感情労働も重要な役割である。さらに、明確な担当者がいない「隙間の仕事」や、問題が発生した際の責任を引き受ける心理的負担も伴う。これらの業務は、工数管理や業績評価の対象になりにくいにもかかわらず、組織を円滑に機能させるために不可欠なものである。

\section{負荷集中の構造的要因}

一人の管理者が直接管理する部下の数を管理スパン(Span of Control)と呼ぶ。Graicunas\cite{graicunas1937}は、管理者と部下の間に生じる関係性の数が管理スパンに対して指数関数的に増加することを数学的に示した。部下が$n$人の場合、管理者が対応すべき関係性の数は$n(2^{n/2} + n - 1)$となる。例えば、部下が5人の場合は100程度の関係性だが、10人になると5,000以上の関係性が生じる計算になる。現代のIT企業では、管理スパンが10人を超えることも珍しくない。この広い管理スパンが、調整負荷の増大を構造的に引き起こしている。

組織をネットワークとして捉えると、中間管理職は上位層と下位層を結ぶ結節点(ハブ)の位置にある。情報は中間管理職を経由して上下に流れるため、中間管理職には情報が集中しやすい構造になっている。この構造は、中間管理職がボトルネックになりやすいことを意味する。中間管理職が処理しきれないほどの情報が集中すると、情報の遅延や欠落が発生し、組織全体のパフォーマンスに影響を与える。

タスクを部下に委譲することで、管理職自身の作業負荷は軽減できる。しかし、委譲には「分割コスト」が伴う。タスクを委譲する際には、背景の説明、期待する成果の明確化、進捗の確認、成果物のレビューと統合など、追加的な作業が発生する。部下の数が多いほど、この分割コストの総量は増大する。また、部下からの報告を受ける際にも、内容の確認、質問への回答、フィードバックの提供といった作業が発生する。これらの「委譲に伴うコスト」は見えにくいが、管理職の負荷を構成する重要な要素である。

\section{本研究の目的}

以上の背景を踏まえ、本研究では以下の研究課題に取り組む。第一に、中間管理職への負荷集中は個人の能力や効率の問題ではなく、組織構造に起因する構造的問題であることを定量的に示す。第二に、管理スパンの変更が負荷分布にどのような影響を与えるかをシミュレーションにより明らかにする。第三に、負荷を軽減するための組織構造改善の方向性を探索する。

本研究では、エージェントベースシミュレーションを用いて組織内のタスクフローをモデル化した。具体的には、5層階層組織(CXO $\rightarrow$ Director $\rightarrow$ SeniorManager $\rightarrow$ Manager $\rightarrow$ Player)をグラフとして表現し、タスクがこのグラフ上を移動・処理される過程をシミュレートした。各ノード(役職)がタスクを受け取り、処理し、委譲し、報告を受ける際に発生する負荷を数値化することで、負荷分布を定量的に評価した。このアプローチにより、組織構造の変更(特に管理スパンの変更)が負荷分布に与える影響を実験的に検証することが可能である。

本研究の学術的意義は、中間管理職の負荷問題を「個人の問題」から「構造の問題」へと再定義し、定量的な分析フレームワークを提供することにある。実務的な意義としては、組織設計者や人事担当者が組織構造の変更を検討する際に、その影響を事前に評価するための基盤を提供することが期待できる。中間管理職の過負荷は、バーンアウトや離職の原因となり、組織の持続可能性を脅かす重大な問題である。本研究が、この問題の構造的理解と解決に向けた一助となることを目指す。
