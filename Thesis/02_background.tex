\chapter{背景と問題設定 (Background)}

\section{中間管理職の負荷増大という問題}

近年、多くの組織において中間管理職の業務負荷が過度に高まっていることが指摘されている。
特に IT 産業や知識集約型産業では、業務の高度化・専門化が急速に進行しており、
個々の業務は単純な作業遂行ではなく、複数の利害関係者との調整や、
状況に応じた判断、文脈を踏まえた意思決定を伴うものへと変化している。

このような変化に加え、近年推進されている「働き方改革」や組織のフラット化、
権限委譲の促進といった組織改革は、必ずしも中間管理職の負担軽減には結びついていない。
むしろ、意思決定の迅速化や現場の自律性向上を目的とした施策の結果として、
現場からの相談・報告・調整業務が中間管理職層に集中する構造が強化されている側面がある。

その結果、中間管理職は上位層から降りてくる戦略的・抽象的な要求と、
現場で発生する具体的かつ即時的な問題対応の双方を同時に引き受ける、
いわゆる「サンドイッチ状態」に置かれやすくなっている。
この状態では、業務内容の頻繁な切り替え、すなわち長時間にわたる
コンテキストスイッチが常態化し、一つ一つの業務に十分な時間や認知資源を割くことが困難となる。

\section{現場で観測される症状}

現場レベルでは、このような構造的状況に起因する複数の症状が報告されている。
中間管理職は自身が直接手を動かす作業時間をほとんど確保できず、
調整・判断・説明・承認といった間接的かつ不可視な業務に多くの時間を費やす傾向が強い。
また、関係者の増加に伴い、単一のタスクであっても複数部門や複数人との
合意形成が必要となり、タスクの複雑度および認知的負荷が急激に増大する。

さらに、中間管理職は正式な業務フローに明示されない
「隙間の仕事」、すなわち責任の所在が曖昧な調整や突発的な相談、
非公式な意思決定支援を引き受けることが多い。
これらの業務は業績評価や工数管理の対象になりにくい一方で、
心理的・精神的負担は大きく、燃え尽き症候群(Burnout)のリスクを高める要因となっている。

\section{既存研究の限界と研究課題}

従来の研究や実務的議論においては、
中間管理職の過重負担は個人の能力不足やマネジメントスキルの欠如といった
個人要因に帰着されることが多かった。
しかし、同様の問題が業界や組織を横断して広く観測されている現状を踏まえると、
負荷の集中は個人の資質のみでは説明しきれない。

むしろ、組織の階層構造やレポートライン、役割分担の設計、
さらにはタスクがどのような経路で生成・分配・処理されるかという
タスクフローの構造そのものが、
中間管理職の負荷を構造的に増幅している可能性が高いと考えられる。

\section{本研究のアプローチ}

本研究では、中間管理職の過重負担を個人の問題としてではなく、
組織構造およびタスクフローに起因する構造的問題として捉える立場を取る。
具体的には、組織をネットワークとしてモデル化し、
階層構造とタスクの流れが各役職層にどのような負荷分布をもたらすのかを
定量的に評価することを目的とする。

特に、タスクがどの階層で生成され、
どの経路を通って移動し、
どの役職層で分割・調整・判断されるのかに着目し、
エージェントベースシミュレーションおよびグラフ理論的指標を用いて分析を行う。
これにより、「なぜ中間管理職の負荷が高くなりやすいのか」という問いに対し、
構造とアルゴリズムの観点から説明可能な枠組みを提示することを目指す。
