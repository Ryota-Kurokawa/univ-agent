\section*{研究背景と目的}
現代の知識集約型企業において、中間管理職への業務負荷が過度に高まっている。階層型組織構造では、中間管理職が双方向の情報処理を担うため、情報フローのボトルネックとなりやすい\cite{mizuta2008}。本研究では組織構造とタスクフローに起因する構造的問題として捉え、エージェントベースシミュレーションを用いて定量的に分析した。研究目的は、(1) 中間管理職への負荷集中の定量的確認、(2) 管理スパン変更が負荷分布に与える影響の解明、(3) 負荷軽減のための組織構造最適化手法の提案である。

\section*{提案手法}
5層階層組織(CXO → Director(Dir) → SeniorManager(SM) → Manager(Mgr) → Player、総ノード数約25,000)をグラフとしてモデル化し、TaskAgentがタスクを表現して組織内を移動・処理される過程をシミュレートした。タスクは上位層で生成され、中間層で分割・委譲され、下位層で実行される。実行後は報告が上位へ集約される。各ノードには、タスク受信・送信・報告受信・分割の各操作に応じた負荷が蓄積される。パラメータはIT企業3名へのインタビュー調査に基づいて設定し、2000ステップのシミュレーションを実行した。

\section*{実験設計}
中間管理職の負荷を軽減する2つのアプローチを比較検証した。

\textbf{実験A(効率化):}1人あたりの処理効率向上を目指した再構成を行う。具体的にはSeniorManager人数を238名に固定し、SM→Mgr管理スパンを削減([10,15]→[4,6])、Mgr→Player管理スパンを増加([5,10]→[18,22])。

\textbf{実験B(分散化):}総負荷を分散させることを目指した再構成を行う。具体的にはDir→SM管理スパンを増加([10,15]→[18,23])してSM人数を増やし(238名→398名)、SM→Mgr管理スパンを削減([10,15]→[4,6])。

両実験とも4段階(Step1~4)で段階的に管理スパンを変更し、InterviewBasedシナリオをベースラインとした。

\section*{主要な結果}
\begin{table}[h]
\centering
\small
\begin{tabular}{lcc}
\toprule
& Base & A \\
\midrule
SM人数 & 238 & 238 \\
SM平均SimLoad & 94,420 & 81,562 (-14\%) \\
SM正規化SimLoad & 2.67 & 2.19 (-18\%) \\
Mgr平均タスク数 & 56,625 & 139,376 (+146\%) \\
\bottomrule
\end{tabular}
\quad
\begin{tabular}{lcc}
\toprule
& Base & B \\
\midrule
SM人数 & 238 & 398 \\
SM平均SimLoad & 94,420 & 74,369 (-21\%) \\
SM正規化SimLoad & 2.67 & 3.21 (+20\%) \\
Mgr平均タスク数 & 56,625 & 83,313 (+47\%) \\
\bottomrule
\end{tabular}
\end{table}

両実験ともSM→Mgrスパン削減によりSMの絶対負荷は減少し、管理スパン削減の有効性を確認した。実験AではSMの処理効率が18\%向上したが、Mgrのタスク数が146\%増加した。実験BではSM絶対負荷が21\%減少したが、1タスクあたり負荷は20\%増加し効率は低下した。

\section*{考察と結論}
本研究により、中間管理職への負荷集中は組織構造に起因する構造的問題であることを定量的に示した。管理スパンの向上には本質的なトレードオフが存在し、実験Aは「Mgrへの負荷遷移」、実験Bは「1人あたり効率の低下」という異なるトレードオフを持つ。適切な管理スパン範囲は5-6人程度でGraicunasの理論\cite{graicunas1937}と整合した。本研究の意義は、組織改変の程度と負荷移動の関係を定量的に示したことにある。これにより組織設計者は組織構造変更の影響を事前に予測し、科学的根拠に基づいた意思決定が可能となる。

\section*{参考文献}
\begin{thebibliography}{9}

\bibitem{mizuta2008}
水田秀行, 企業組織ネットワークの解析 ~戦略的な組織構造と個人間のコミュニケーションの役割~, 情報処理 49 (2008) 298--303.

\bibitem{graicunas1937}
Graicunas, V. A., Relationship in Organization, Papers on the Science of Administration (1937) 181--187.

\end{thebibliography}
