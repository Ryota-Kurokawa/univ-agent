\section*{研究背景と目的}
現代の知識集約型企業において、中間管理職への業務負荷が過度に高まっている。階層型組織構造では、中間管理職が上位からの戦略的指示の伝達と下位からの業務報告の集約という双方向の情報処理を担うため、情報フローのボトルネックとなりやすい\cite{mizuta2008}。この問題は従来、個人の能力不足として扱われてきたが、本研究では組織構造とタスクフローに起因する構造的問題として捉え、エージェントベースシミュレーションを用いて定量的に分析した。本研究の目的は、(1) 中間管理職への負荷集中の定量的確認、(2) 管理スパン変更が負荷分布に与える影響の解明、(3) 負荷軽減のための組織構造最適化手法の提案である。

\section*{提案手法}
5層階層組織(CXO → Director(Dir) → SeniorManager(SM) → Manager(Mgr) → Player、総ノード数約25,000)をグラフとしてモデル化し、TaskAgentがタスクを表現して組織内を移動・処理される過程をシミュレートした。タスクは上位層で生成され、中間層で分割・委譲され、下位層で実行される。実行後は報告が上位へ集約される。各ノードには、タスク受信・送信・報告受信・分割の各操作に応じた負荷が蓄積される。パラメータはIT企業3名へのインタビュー調査に基づいて設定し、2000ステップのシミュレーションを実行した。

\section*{実験設計}
中間管理職の負荷を軽減する2つのアプローチを比較検証した。

\textbf{実験A(効率化):}SeniorManager人数を238名に固定し、SM→Mgr管理スパンを削減([10,15]→[4,6])、Mgr→Player管理スパンを増加([5,10]→[18,22])。狙いは1人あたりの処理効率向上。

\textbf{実験B(分散化):}Dir→SM管理スパンを増加([10,15]→[18,23])してSM人数を増やし(238名→398名)、SM→Mgr管理スパンを削減([10,15]→[4,6])。狙いは総負荷の分散。

両実験とも4段階(Step1~4)で段階的に管理スパンを変更し、InterviewBasedシナリオをベースラインとした。

\section*{主要な結果}
\begin{table}[h]
\centering
\small
\begin{tabular}{lccc}
\toprule
& Baseline & 実験A(Step4) & 実験B(Step4) \\
\midrule
SM人数 & 238名 & 238名 & 398名 \\
SM平均SimLoad & 94,420 & 81,562(-14\%) & 74,369(-21\%) \\
SM正規化SimLoad & 2.67 & 2.19(-18\%) & 3.21(+20\%) \\
Mgr平均タスク数 & 56,625 & 139,376(+146\%) & 83,313(+47\%) \\
\bottomrule
\end{tabular}
\end{table}

第一に、両実験ともSM→Mgrスパン削減によりSMの絶対負荷は減少し、管理スパン削減の有効性を確認した。第二に、実験AではSMの1タスクあたり負荷が18\%減少し処理効率が向上したが、Mgrのタスク数が146\%増加した。第三に、実験BではSM人数が67\%増加し絶対負荷は21\%減少したが、1タスクあたり負荷は20\%増加し効率は低下した。第四に、両実験のStep4でSMとMgrの最高SimLoadがほぼ均衡し(実験A: 90,780 vs 84,767、実験B: 82,670 vs 82,174)、負荷均衡点の存在を確認した。

\section*{考察と結論}
本研究により、中間管理職への負荷集中は組織構造に起因する構造的問題であることを定量的に示した。管理スパン最適化には本質的なトレードオフが存在し、実験Aは「Mgrへの負荷遷移」、実験Bは「1人あたり効率の低下」という異なるトレードオフを持つ。適切な管理スパンの範囲は5-6人程度であることを確認し、Graicunasの理論\cite{graicunas1937}と整合する結果を得た。組織の目標(効率重視か負荷分散重視か)に応じて適切なアプローチを選択すべきである。本研究の貢献は、組織設計者が組織構造の変更を検討する際に、科学的根拠に基づいた意思決定を行うための定量的フレームワークを提供したことにある。

\section*{参考文献}
\begin{thebibliography}{9}

\bibitem{mizuta2008}
水田秀行, 企業組織ネットワークの解析 ~戦略的な組織構造と個人間のコミュニケーションの役割~, 情報処理 49 (2008) 298--303.

\bibitem{graicunas1937}
Graicunas, V. A., Relationship in Organization, Papers on the Science of Administration (1937) 181--187.

\end{thebibliography}
