\chapter{概要}

\section{研究背景}

現代の知識集約型産業において、中間管理職の業務負荷が過度に高まっている。中間管理職は上位層からの戦略的指示と現場からの報告・相談を同時に処理する「サンドイッチ状態」に置かれ、タスク分割、調整、報告受信といった不可視の業務が集中する構造である。従来、この問題は個人の能力不足として扱われることが多かったが、本研究では組織構造とタスクフローに起因する構造的問題として捉え、定量的な分析を試みた。

\section{提案手法}

本研究では、エージェントベースシミュレーションとグラフ理論を統合した組織負荷分析フレームワークを提案する。5層階層組織(CXO $\rightarrow$ Director $\rightarrow$ SeniorManager(以下SM) $\rightarrow$ Manager(以下Mgr) $\rightarrow$ Player)をグラフとしてモデル化し、TaskAgentがタスクを表現して組織内を移動・処理される過程をシミュレートした。各ノードのSimLoad(シミュレーション負荷)を定量的に計測し、管理スパン(Span of Control)を段階的に変更することで負荷遷移を観察した。

\section{実験設計}

IT企業へのインタビュー調査に基づく InterviewBased シナリオをベースラインとし、中間管理職の負荷を軽減する 2 つのアプローチを比較検証した。全シナリオで総ノード数を約 25,000 に維持し、組織規模の影響を排除した。実験A(効率化)ではSM人数を238名に固定したまま管理スパンを削減し、実験B(分散化)ではSM人数を238名から398名へ増加させて総負荷を分散する設計とした。

\section{主要な結果}

第一に、両実験ともSM→Mgrスパンを削減することでSMの絶対負荷は減少し(実験A: 14\%減、実験B: 21\%減)、管理スパン削減が有効であることを示した。第二に、効率化アプローチ(実験A)ではSMの1タスクあたり負荷が18\%減少し処理効率が向上した。ただしMgrのタスク数が146\%増加し、Mgrへの負荷遷移が大きいことを明らかにした。第三に、分散化アプローチ(実験B)ではSM人数が67\%増加し絶対負荷は効率化より大きく減少した。しかし1タスクあたり負荷は20\%増加し、効率は低下した。第四に、効率化は「Mgrへの負荷遷移」、分散化は「1人あたり効率の低下」という異なるトレードオフを持つことを明らかにした。

\section{結論}

本研究により、以下のことを定量的に示した。中間管理職への負荷集中は組織構造に起因する構造的問題であり、管理スパン削減は負荷軽減に効果的である。適切な管理スパンの範囲は5-6人程度であることを確認した。負荷軽減には「効率化」と「分散化」の2つのアプローチがあり、それぞれ異なるトレードオフを持つため、組織の目標(効率重視か負荷分散重視か)に応じて適切なアプローチを選択すべきである。
