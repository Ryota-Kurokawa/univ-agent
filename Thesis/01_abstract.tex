\chapter{概要 (Abstract)}

\section{序論:若年層の昇進忌避と現代組織の構造的限界}
現代の先進諸国、とりわけ日本を含む成熟社会の企業組織において、若年層を中心とした「管理職への昇進忌避」という現象が進行している。近年の労働意識調査が示す通り、多くの若手従業員は地位や権限の獲得よりも、現状維持や専門職としてのキャリアを志向する傾向が顕著である。
著者は、この現象を単なる「労働意欲の低下」や「野心の欠如」といった個人の資質やマインドセットの問題に帰結させるべきではないと考える。これは、現代の知識集約型産業において、中間管理職(マネージャー、リーダー、OSSにおけるメンテナー等)に求められる業務が、人間の認知限界を超えて肥大化・複雑化していることに対する、極めて合理的な防衛反応であると捉えるべきである。

かつてのピラミッド型組織における管理職は「命令と統制」が主な役割であったが、ソフトウェアエンジニアの分野に絞っても、アジャイル開発やGitHub等を用いた分散型開発が主流となった現代において、その役割は「情報のハブ」へと変容した。そこでは、多岐にわたるステークホルダー間の利害調整、開発コンテキストの維持、メンタルヘルスケア、そして技術的な意思決定といった多次元的なタスクが、特定の中間ノードに集中する構造となっている。この「責任の無限拡大」と「権限・リソースの限定性」の構造的不均衡こそが、若者を出世から遠ざける根本原因であり、組織を持続不可能にしている致命的な欠陥であると考える。

\section{課題の再定義:コミュニケーションコストとタスク摩擦の抽象化}
本研究では、この複雑怪奇な組織課題を解決するために、個人の努力や精神論に頼るのではなく、工学的かつ数学的なアプローチを採用する。その第一歩として、「人の負荷(Load)」という概念を再定義する。
従来、業務負荷は「作業時間」や「タスク数」で定量化されてきた。しかし、現代のナレッジワーカーを真に疲弊させているのは、単純な作業量だけではなく、タスクAからタスクBへと注意を切り替える際のスイッチングコスト(認知資源の消耗)や、人から人へとタスクを受け渡す際に生じるコミュニケーションの「摩擦」も含まれるのではないだろうか。

著者は、組織を巨大な情報処理ネットワークと捉え、これらの不可視の負荷をネットワーク上の「流量」や「重み」として抽象化することを試みる。人間関係、報告ライン、業務フローといった定性的な要素を、グラフ理論における「ノード(主体)」と「エッジ(関係性)」に単純化してマッピングすることで、これまで感覚的にしか語られなかった「組織の構造的ストレス」を、客観的かつ操作可能な数値として扱うためのフレームワークを構築する。

\section{提案手法:グラフ理論に基づく動的負荷分散モデル}
本研究の核心は、組織構造とタスクフローを統合した「動的な多重グラフ(Dynamic Multigraph)モデル」の提案にある。このモデルにおいて、中間管理職は上司と部下をもち、その間でコンテキストのスイッチにおけるスイッチングコストやコミュニケーション、受け取ったタスクのスプリット作業などを担当する役割を持つ。また、一定の媒介中心性の高さを属性として持ち合わせており、それが高いということは、そのノードが除去されると組織内の情報伝達が分断される、あるいは極端に効率が低下することを意味し、これこそが「負荷集中」のグラフ理論的正体であると考えられる。

既存の研究でも組織のグラフ化は試みられているが、多くは人事発令に基づく静的な組織図の解析に留まっていた。しかし現実の組織は、日々の業務依頼、相談、承認といった「情報の流量」によって絶えず形を変える生き物のような存在である。
本研究では、こうした動的なコミュニケーションの発生状況に応じて、人と人との結合強度がリアルタイムに変動する動的モデルを採用する。特定のノードに情報や権限が過度に集中し、閾値を超えた場合、そのつながり(エッジ)を動的に組み替えるアルゴリズムを用いる。

\section{結論}
本研究では、提案手法を実際のオープンソースプロジェクトのデータに適用し、エージェント導入前後でのネットワーク指標の変化を検証した。その結果、中間管理ノードの媒介中心性が有意に低下し、組織全体のタスク処理効率(平均パス長による評価)が維持・向上することを確認した。
本論文は、「若者の管理職離れ」という社会学的課題に対し、グラフ理論による「組織の幾何学的最適化」と、AIエージェントによる「労働の代替」という二つの側面から包括的な解を提示するものである。人間が「調整」や「伝書鳩」のようなハブ機能から解放され、より創造的で人間的な意思決定に注力できる未来の組織像を示すことは、人とAIが協調する次世代の組織デザイン(Organizational Design)における重要なマイルストーンとなる。