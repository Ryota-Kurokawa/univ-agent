\chapter{考察と結論}
\label{chap:discussion}

\section{研究結果の要約}

本研究では、中間管理職への負荷集中問題を組織構造とタスクフローに起因する構造的問題として捉え、エージェントベースシミュレーションを用いた定量的分析を行った。具体的には、5層階層組織(CXO $\rightarrow$ Director $\rightarrow$ SeniorManager $\rightarrow$ Manager $\rightarrow$ Player)をグラフとしてモデル化し、TaskAgent がタスクを表現し、組織内を移動・処理される過程をシミュレートした。各ノードの負荷(SimLoad)を定量的に計測することで、構造的な負荷集中のメカニズムを明らかにした。

本研究では、2 つの異なるアプローチで管理スパンの変更実験を行った。\textbf{実験 A} では SM→Mgr 管理スパン削減と Mgr→Player 管理スパン増加を実施し、\textbf{実験 B} では Dir→SM 管理スパン増加と SM→Mgr 管理スパン削減を実施した。

主要な発見は以下の通りである。第一に、\textbf{中間管理職への負荷集中の定量的確認}として、InterviewBased シナリオにおいて、SeniorManager 層の平均 SimLoad(94,420)が他の役職層と比較して最も高いことを確認した。これは、SeniorManager が上位からの戦略的指示を受け、複数の Manager にタスク分割し、さらに報告を受け取るという多重の役割を担っていることに起因する。第二に、\textbf{実験 A による負荷軽減効果}として、SM→Mgr 管理スパンを削減することで、SeniorManager の絶対的負荷と正規化負荷(1 タスクあたり)の両方が減少した。Step4 では、SM の平均 SimLoad は 94,420 から 81,562 へ約 14\% 減少し、正規化 SimLoad は 2.67 から 2.19 へ約 18\% 減少した。第三に、\textbf{実験 B による負荷分散効果}として、Dir→SM 管理スパンを増加させることで、SeniorManager の人数を 238 名から 398 名へ増やし、絶対的負荷を分散できた(平均 94,420 → 74,369、約 21\% 減少)。ただし、正規化負荷は 2.67 から 3.21 へ約 20\% 増加し、1 タスクあたりの処理効率は低下した。第四に、\textbf{負荷遷移パターンの違い}として、実験 A では Mgr のタスク数が大幅増加(56,625 → 139,376)し、実験 B では SM の人数増加(238 → 398)で負荷を吸収した。両アプローチとも Step4 時点で SM と Mgr の最高 SimLoad がほぼ均衡した。

\section{考察}

本研究の結果は、中間管理職への負荷集中が個人の能力や効率の問題ではなく、組織構造とタスクフローに起因する構造的問題であることを示唆している。シミュレーション結果から、負荷集中を引き起こす主要なメカニズムとして以下の3点が挙げられた。第一にタスク分割コストである。中間管理職は、上位から受け取ったタスクを複数の部下に分割する際に、各部下への説明、スケジュールの調整、成果物の統合といった追加的な負荷が発生する。本研究では、この分割コストをタスク重みの 20\% として明示的にモデル化した。管理スパンが広い(部下の数が多い)ほど、分割コストの総量が増加し、管理職の負荷が高まる。これは、単純にタスクを委譲するだけでなく、「委譲すること自体が負荷である」という現実を反映している。第二に報告受信コストである。部下からの報告を受け取る際にも、内容の確認、質問への回答、といった負荷が発生する。本研究では、この報告コストをタスク重みの 10\% として反映した。部下の数が多いほど、報告の総量が増加し、管理職の負荷が高まる。特に、管理スパンが広い場合、複数の報告が集中して管理する必要があり、認知的負荷が高まる傾向がある。第三にエスカレーション処理である。部下が処理しきれない困難なタスクは、上位へエスカレーションされる。本研究では、タスクの難易度が閾値を超えた場合、5\% の確率でエスカレーションが発生するモデルを採用した。中間管理職は、その位置から部下からのエスカレーションと、上位からの指示の両方を受け取るため、組織の「交差点」となりやすい。これが、中間管理職の負荷集中を構造的に引き起こす要因である。

本研究では、管理スパン最適化に対する 2 つの異なるアプローチを検証した。\textbf{実験 A(SM→Mgr 削減)}では、SeniorManager の管理スパンを削減し、Manager の管理スパンを増加させた。この結果、SM の絶対的負荷は 94,420 から 81,562 へ 14\% 減少し、SM の正規化負荷は 2.67 から 2.19 へ 18\% 減少し、Mgr の平均タスク数は 56,625 から 139,376 へ 146\% 増加した。このアプローチは、SM の 1 タスクあたりの処理効率を向上させる効果がある。SM の管理スパン削減により、タスク分割・報告受信の回数が減少し、各タスクに集中できる環境が整う。ただし、Mgr のタスク数が大幅に増加するため、Mgr 層への負荷遷移が発生する。

一方、\textbf{実験 B(Dir→SM 増加)}では、Director の管理スパンを増加させ、SeniorManager の人数を増やした。この結果、SM の人数は 238 から 398 へ 67\% 増加し、SM の絶対的負荷は 94,420 から 74,369 へ 21\% 減少し、SM の正規化負荷は 2.67 から 3.21 へ 20\% 増加した。このアプローチは、SM の人数増加により総負荷を分散する効果がある。しかし、各 SM の管理スパンが狭くなることで、委譲先が減少し、自身で処理すべきタスクの割合が増加する。結果として、1 タスクあたりの処理効率は低下する。

両実験の比較から、管理スパン最適化には本質的なトレードオフが存在することが明らかになった。実験 A では SM 人数固定で効率向上を実現するが Mgr への負荷遷移が発生し、実験 B では SM 人数増加で負荷分散を実現するが 1 人あたり効率が低下する。この結果は、「組織の総業務量は一定であり、特定の役職層の負荷だけを削減することは困難である」という組織設計の本質的な制約を示している。実験結果から、管理スパンの最適化は組織設計において重要であることを改めて確認した。しかし、単純に管理スパンを削減すればよいわけではなく、様々なトレードオフの考慮が必要である。両実験とも、Step4 時点で SM と Mgr の最高 SimLoad がほぼ同等となった。実験 A では SM 最高 90,780 vs Mgr 最高 84,767、実験 B では SM 最高 82,670 vs Mgr 最高 82,174 であった。この「負荷均衡点」の存在は、管理スパンの調整により特定役職への過度な負荷集中を緩和できることを示唆している。ただし、均衡点を超えて調整を続けると、負荷の逆転(Mgr > SM)が発生する可能性がある。古典的な組織理論では、適切な管理スパンは 5-6 人程度とされてきた。本研究の結果は、現代の IT 企業においても、この範囲が妥当であることを示唆している。InterviewBased シナリオでは、SeniorManager の管理スパンが平均 12.5 人であり、これが過大な負荷を引き起こしていた。実験 A の Step4(平均 5.0 人)では絶対・正規化負荷ともに減少し、管理スパン 5-6 人が効率的であることを確認した。組織の目標に応じて、適切なアプローチを選択する必要がある。1 人あたりの効率向上を重視する場合は実験 A(SM→Mgr 削減)が適しており、総負荷の分散と人員確保を重視する場合は実験 B(Dir→SM 増加)が適している。両方のバランスを取る場合は、両アプローチの組み合わせを検討すべきである。本研究の結果は、実務上の組織設計に対して以下の示唆を提供する。中間管理職の過負荷を個人の能力問題と見なすのではなく、組織構造とタスクフローに起因する構造的問題として認識する必要がある。本研究では、SM の管理スパンが平均 12.5 人の場合、平均 SimLoad が 94,420 となり、組織内で最も高い負荷が発生することを定量的に示した。本研究で検証した 2 つのアプローチは、それぞれ異なる状況に適している。実験 A 型(SM→Mgr 削減)は、SM 人数を維持しつつ効率化したい場合に有効であり、SM の 1 タスクあたり負荷を 18\% 削減できるが、Mgr のタスク数が 2.5 倍に増加するため、Mgr 層に余裕がある組織に適している。一方、実験 B 型(Dir→SM 増加)は、SM のポジションを増やしたい場合に有効であり、SM 人数を 67\% 増加でき昇進機会の拡大につながるが、1 人あたりの処理効率は 20\% 低下するため、業務の質よりも負荷分散を優先する場合に適している。両実験で観測した「負荷均衡点」(SM と Mgr の最高 SimLoad が同等となる点)は、組織設計の目標として有用である。Step4 時点での均衡点を参考に、SM→Mgr 管理スパンを 4-6 人程度に設定することで、特定役職への過度な負荷集中を緩和できる。組織の負荷状況を評価する際は、絶対値(総 SimLoad)と正規化値(SimLoad / タスク数)の両面から分析することが重要である。実験 B では、絶対値は 21\% 減少したが正規化値は 20\% 増加しており、単一指標だけでは組織状態を正確に把握できない。SeniorManager の負荷軽減のために新たな役職層(SubManager 等)を追加するよりも、既存の役職層間の管理スパンを調整する方が、意思決定の遅延や情報伝達の歪みを避けられる。本研究の実験 A・B は、役職層を追加せずに負荷を再分配する方法を示している。本研究では、タスクフローを役職ごとに固定したが、実際の組織では、タスクフローの柔軟な見直しも可能である。例えば、上位からの Top-down タスクを減少し、現場からの Design タスクを増やすことで、負荷分散をより適切に行える可能性がある。

\section{限界と今後の課題}

本研究にはいくつかの限界がある。本研究では、組織を 5 層階層のツリー構造としてモデル化したが、実際の組織はより複雑な構造を持つ。例えば、マトリクス型組織やプロジェクトベースの組織では、複数の指揮系統や横断的なコミュニケーションが存在する。また、タスクの動的現実性も限定されている。実際の組織では、タスクの優先度変更や中断・再開、外部要因による変動などが日常茶飯事であり、これらを正確にモデル化することで、より現実的なシミュレーションが可能となる。シミュレーションのパラメータ(タスクフロー確率、分割コスト、報告コストなど)は、限定的なインタビューに基づいて設定した。追加的な実証研究を通じて、これらのパラメータの妥当性を検証する必要がある。特に、分割コスト(20\%)や報告コスト(10\%)といった数値は、業種や業界によって大きく変わる可能性がある。本研究では、固定された組織構造の中でシミュレーションしたが、実際の組織はあらゆる要因により変化する。人員の増減、役職の変更、チーム再編成など、動的な変化をモデル化することで、より現実的な分析が可能となる。今後の研究として、以下の方向性が考えられる。実際の企業の組織データやタスクログを用いて、シミュレーション結果との比較検証を行う。特に、本研究で発見した「負荷均衡点」が実組織でも観測されるかを検証する。実験 A と実験 B の組み合わせにより、絶対的負荷と正規化負荷の両方を改善できる最適な管理スパン設定を探索する。多目的最適化手法の適用が有効と考えられる。マトリクス型組織やネットワーク型組織など、より多様な構造に適用した分析を実施する。非階層的な組織では、管理スパンの概念自体が異なる可能性がある。人員の増減や役職変更など、組織の動的な変化をモデル化し、組織成長過程における負荷遷移をシミュレーションする。中間管理職の業務を補助する AI エージェントを導入した場合の負荷変化をシミュレーションによって分析する。特に、タスク分割や報告受信といった調整コストの削減効果を検証する。

\section{結論}

本研究では、中間管理職の負荷集中問題に対して、エージェントベースシミュレーションとグラフ理論を組み合わせた新たな分析フレームワークを提示した。実務者インタビューに基づくシナリオ設計を行い、5 層階層組織におけるタスクフローシミュレーションを実施することで、中間管理職への負荷集中メカニズムを定量的に分析した。本研究の主要な成果は以下の通りである。負荷集中の定量化として、InterviewBased シナリオにおいて、SeniorManager 層の平均 SimLoad が 94,420 と最も高く、中間管理職への負荷集中を定量的に確認した。2 つの最適化アプローチの提示として、管理スパン最適化に対して、実験 A(SM→Mgr 削減)と実験 B(Dir→SM 増加)の 2 つのアプローチを検証し、それぞれの効果とトレードオフを明らかにした。負荷均衡点の発見として、両実験で、Step4 時点において SM と Mgr の最高 SimLoad がほぼ均衡する点を観測した。この均衡点は、組織設計の目標として活用できる。絶対値と正規化値の乖離として、実験 B では絶対的負荷が 21\% 減少する一方、正規化負荷が 20\% 増加することを発見し、組織評価には複数の指標が必要であることを示した。本研究の貢献は、組織設計者や経営者が組織構造の変更を検討する際に、事前に科学的に基づいた意思決定を行うための基盤を提供することである。特に、「SM の効率向上」と「SM の人数増加による負荷分散」という 2 つの選択肢を定量的に比較できる枠組みを提示した。中間管理職の負荷問題は、現代の知識集約型企業において重要な課題であり、単なる管理職個人への育成や組織の持続可能性に関わる問題である。本研究が、この問題の構造的理解と、科学に基づいた組織設計の発展に貢献できれば幸いである。今後、より現実的な組織構造やタスクフローの特徴を反映した分析の拡張、実証データを用いた検証、両アプローチの組み合わせによる最適解の探索など、さらなる研究の発展が期待される。中間管理職が健全で持続可能な働き方ができる組織の実現に向けて、本研究が一つの道標となることを願っている。
