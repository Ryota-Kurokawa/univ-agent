\chapter{評価結果}
\label{chap:results}

本章では、第 3 章で設計した管理スパン変更実験の結果を報告する。「中間管理職の負荷を構造的に軽減するにはどうすればよいか?」という問いに対して、2 つの異なるアプローチを比較検証した。第一は\textbf{実験 A(効率化アプローチ)}であり、SM 人数を固定し各 SM の管理スパンを削減する。第二は\textbf{実験 B(分散化アプローチ)}であり、SM 人数を増やし総負荷を分散する。両実験とも InterviewBased シナリオをベースラインとし、総ノード数を約 25,000 に維持している。これにより、組織規模の影響を排除した上で、2 つのアプローチの効果を直接比較できる。

\section{実験設計とベースライン}

本研究では、InterviewBased シナリオをベースラインとして、2 つの異なる管理スパン変更実験を実施する。InterviewBased シナリオは、実際の IT 企業へのインタビュー調査に基づいて設定した現実的な組織構造を反映している。管理スパンは、CXO が 15-20 人(平均 17.5)、Director が 10-15 人(平均 12.5)、SeniorManager が 10-15 人(平均 12.5)、Manager が 5-10 人(平均 7.5)である。

総ノード数は 25,731 名であり、CXO が 1 名、Director が 20 名、SeniorManager が 238 名、Manager が 2,976 名、Player が 22,496 名から構成される。

役職層別の平均 SimLoad は、CXO が 47,789、Director が 70,990、SeniorManager が 94,420(最大 106,551)、Manager が 64,903(最大 73,839)、Player が 7,636 である。SeniorManager 層の平均 SimLoad が最も高く、中間管理職への負荷集中を観測した。このベースラインに対して、実験 A と実験 B の 2 つのアプローチで管理スパンを変更し、負荷分布の変化を評価した。

\section{実験 A:効率化アプローチ}

本実験では、SeniorManager の人数を固定(238 名)したまま、各 SM の管理スパンを削減する。削減分のタスク処理能力は Manager 層で補うため、Mgr→Player スパンを増加させる。狙いは「1 人あたりの処理効率を向上させる」ことである。

\begin{table}[htbp]
\centering
\caption{実験 A のパラメータ設定}
\label{tab:exp_a_params}
\begin{tabular}{lcccc}
\toprule
シナリオ & SM→Mgr & 変化率 & Mgr→Player & 変化率 \\
\midrule
InterviewBased & [10, 15] & --- & [5, 10] & --- \\
Step1 & [8, 11] & $-24\%$ & [8, 12] & $+33\%$ \\
Step2 & [6, 9] & $-40\%$ & [12, 16] & $+87\%$ \\
Step3 & [5, 7] & $-52\%$ & [15, 19] & $+127\%$ \\
Step4 & [4, 6] & $-60\%$ & [18, 22] & $+167\%$ \\
\bottomrule
\end{tabular}
\end{table}

Table \ref{tab:exp_a_params} は、各ステップにおける管理スパンのパラメータ設定を示している。SM→Mgr スパンを段階的に削減し、Step4 では 60\% の削減を達成している。

\begin{table}[htbp]
\centering
\caption{実験 A の SimLoad 結果}
\label{tab:exp_a_results}
\begin{tabular}{lrrrrr}
\toprule
シナリオ & 総ノード数 & SM 人数 & SM 平均 & Mgr 人数 & Mgr 平均 \\
\midrule
InterviewBased & 25,731 & 238 & 94,420 & 2,976 & 64,903 \\
Step1 & 25,010 & 238 & 88,624 & 2,240 & 67,234 \\
Step2 & 26,779 & 238 & 86,263 & 1,764 & 70,200 \\
Step3 & 26,042 & 238 & 83,691 & 1,430 & 73,008 \\
Step4 & 25,322 & 238 & 81,562 & 1,192 & 75,961 \\
\bottomrule
\end{tabular}
\end{table}

Table \ref{tab:exp_a_results} より、総ノード数は全シナリオで 25,000〜27,000 の範囲に維持されている。SeniorManager の平均 SimLoad は 94,420 から 81,562 へ約 14\% 減少し、Manager の平均 SimLoad は 64,903 から 75,961 へ約 17\% 増加した。

\begin{table}[htbp]
\centering
\caption{実験 A の正規化 SimLoad}
\label{tab:exp_a_normalized}
\begin{tabular}{lrrrr}
\toprule
シナリオ & SM 平均タスク & Mgr 平均タスク & SM 正規化 & Mgr 正規化 \\
\midrule
InterviewBased & 35,406 & 56,625 & 2.67 & 1.15 \\
Step1 & 35,345 & 72,986 & 2.51 & 0.92 \\
Step2 & 38,457 & 99,355 & 2.24 & 0.71 \\
Step3 & 38,067 & 119,697 & 2.20 & 0.61 \\
Step4 & 37,318 & 139,376 & 2.19 & 0.55 \\
\bottomrule
\end{tabular}
\end{table}

Table \ref{tab:exp_a_normalized} の正規化 SimLoad(平均 SimLoad / 平均タスク数)では、SeniorManager は 2.67 から 2.19 へ約 18\% 減少し、Manager は 1.15 から 0.55 へ約 52\% 減少した。これは、両役職とも 1 タスクあたりの負荷が軽減されていることを示す。図 \ref{fig:exp_a_transition}(章末に掲載)より、SeniorManager の最高 SimLoad は 106,551 から 90,780 へ約 15\% 減少し、Manager の最高 SimLoad は 73,839 から 84,767 へ約 15\% 増加した。正規化 SimLoad では、両役職とも減少傾向を示している。

\section{実験 B:分散化アプローチ}

本実験では、Director→SM スパンを増加させ、SeniorManager の人数を増やす(238 → 398 名)。同時に各 SM の管理スパンも削減し、1 人あたりの負担を軽減する。狙いは「総負荷を多くの SM で分散する」ことである。総ノード数を維持するため、Mgr→Player のスパンも調整している。

\begin{table}[htbp]
\centering
\caption{実験 B のパラメータ設定}
\label{tab:exp_b_params}
\begin{tabular}{lccccc}
\toprule
シナリオ & Dir→SM & 変化率 & SM→Mgr & 変化率 & Mgr→Player \\
\midrule
InterviewBased & [10, 15] & --- & [10, 15] & --- & [5, 10] \\
Dir\_Step1 & [12, 17] & $+16\%$ & [8, 11] & $-24\%$ & [7, 10] \\
Dir\_Step2 & [14, 19] & $+32\%$ & [6, 9] & $-40\%$ & [8, 11] \\
Dir\_Step3 & [16, 21] & $+48\%$ & [5, 7] & $-52\%$ & [9, 12] \\
Dir\_Step4 & [18, 23] & $+64\%$ & [4, 6] & $-60\%$ & [10, 13] \\
\bottomrule
\end{tabular}
\end{table}

Table \ref{tab:exp_b_params} は、各ステップにおける管理スパンのパラメータ設定を示している。Dir→SM スパンを段階的に増加させ、Step4 では 64\% の増加を達成している。

\begin{table}[htbp]
\centering
\caption{実験 B の SimLoad 結果}
\label{tab:exp_b_results}
\begin{tabular}{lrrrrr}
\toprule
シナリオ & 総ノード数 & SM 人数 & SM 平均 & Mgr 人数 & Mgr 平均 \\
\midrule
InterviewBased & 25,731 & 238 & 94,420 & 2,976 & 64,903 \\
Dir\_Step1 & 25,211 & 278 & 86,139 & 2,611 & 66,789 \\
Dir\_Step2 & 25,129 & 318 & 80,979 & 2,348 & 68,762 \\
Dir\_Step3 & 25,140 & 358 & 77,134 & 2,142 & 70,986 \\
Dir\_Step4 & 25,343 & 398 & 74,369 & 1,985 & 73,176 \\
\bottomrule
\end{tabular}
\end{table}

Table \ref{tab:exp_b_results} より、総ノード数は約 25,000 で安定して維持されている。SeniorManager の人数は 238 名から 398 名へ 67\% 増加し、各 SM の平均負荷は 21\% 減少した。

\begin{table}[htbp]
\centering
\caption{実験 B の正規化 SimLoad}
\label{tab:exp_b_normalized}
\begin{tabular}{lrrrr}
\toprule
シナリオ & SM 平均タスク & Mgr 平均タスク & SM 正規化 & Mgr 正規化 \\
\midrule
InterviewBased & 35,406 & 56,625 & 2.67 & 1.15 \\
Dir\_Step1 & 30,699 & 63,225 & 2.81 & 1.06 \\
Dir\_Step2 & 27,331 & 69,589 & 2.96 & 0.99 \\
Dir\_Step3 & 25,025 & 76,593 & 3.08 & 0.93 \\
Dir\_Step4 & 23,153 & 83,074 & 3.21 & 0.88 \\
\bottomrule
\end{tabular}
\end{table}

Table \ref{tab:exp_b_normalized} では、SeniorManager の正規化 SimLoad が 2.67 から 3.21 へ \textbf{20\% 増加}しており、実験 A(18\% 減少)とは対照的な傾向を示す。これは、SM の管理スパン削減により委譲先が減少し、1 タスクあたりの処理負荷が増加したことを示唆する。図 \ref{fig:exp_b_transition}(章末に掲載)より、SeniorManager の最高 SimLoad は 106,551 から 82,670 へ約 22\% 減少し、Manager の最高 SimLoad は 73,839 から 82,174 へ約 11\% 増加した。Dir\_Step4 では、SM 最高(82,670)と Mgr 最高(82,174)がほぼ同等となった。正規化 SimLoad では、SeniorManager が顕著に増加している(2.67 → 3.21)。

\section{効率化 vs 分散化:実験間比較}

\subsection{絶対値での比較}

\begin{table}[htbp]
\centering
\caption{実験 A・B の Step4 比較(絶対値)}
\label{tab:comparison_absolute}
\begin{tabular}{lrrrr}
\toprule
指標 & InterviewBased & 実験 A Step4 & 実験 B Step4 & A-B 差 \\
\midrule
総ノード数 & 25,731 & 25,322 & 25,343 & $-21$ \\
SM 人数 & 238 & 238 & 398 & $-160$ \\
SM 平均 SimLoad & 94,420 & 81,562 & 74,369 & $+7,193$ \\
Mgr 人数 & 2,976 & 1,192 & 1,985 & $-793$ \\
Mgr 平均 SimLoad & 64,903 & 75,961 & 73,176 & $+2,785$ \\
\bottomrule
\end{tabular}
\end{table}

Table \ref{tab:comparison_absolute} は、両実験の Step4 時点における絶対値での比較を示している。実験 B は実験 A と比較して以下の特徴がある。第一に、SM の平均負荷をより削減できる(74,369 vs 81,562)。第二に、SM の人数が増加する(398 vs 238)。第三に、Mgr の人数を多く維持できる(1,985 vs 1,192)。

\subsection{正規化値での比較}

\begin{table}[htbp]
\centering
\caption{実験 A・B の Step4 比較(正規化値)}
\label{tab:comparison_normalized}
\begin{tabular}{lrrr}
\toprule
指標 & InterviewBased & 実験 A Step4 & 実験 B Step4 \\
\midrule
SM 正規化 SimLoad & 2.67 & 2.19 & 3.21 \\
Mgr 正規化 SimLoad & 1.15 & 0.55 & 0.88 \\
\bottomrule
\end{tabular}
\end{table}

Table \ref{tab:comparison_normalized} は、正規化 SimLoad の比較を示している。正規化値で見ると、実験 A は SM の 1 タスクあたり負荷を削減できる(2.19)が、実験 B は逆に増加する(3.21)。これは、実験 B では SM の管理スパンが削減されることで、委譲先が減り、各タスクを自身で処理する割合が増加するためと考えられる。

\subsection{特徴比較}

\begin{table}[htbp]
\centering
\caption{実験 A・B の特徴比較}
\label{tab:comparison_features}
\begin{tabular}{lll}
\toprule
指標 & 実験 A(効率化) & 実験 B(分散化) \\
\midrule
SM 人数 & 固定(238 名) & 67\% 増加(238 → 398 名) \\
SM 絶対負荷 & 14\% 減少 & 21\% 減少 \\
SM 正規化負荷 & 18\% 減少 & 20\% 増加 \\
Mgr タスク数 & 146\% 増加 & 47\% 増加 \\
特徴 & 1 人あたり効率向上 & 総負荷分散 \\
トレードオフ & Mgr への負荷遷移 & 1 人あたり効率低下 \\
\bottomrule
\end{tabular}
\end{table}

Table \ref{tab:comparison_features} は、両実験の特徴を比較している。両実験から、「効率化」と「分散化」には明確な違いがあることが明らかになった。\textbf{実験 A(効率化アプローチ)}では、SM 人数を固定(238 名)したまま、SM 絶対負荷は 14\% 減少(94,420 → 81,562)し、SM 正規化負荷も 18\% 減少(2.67 → 2.19)した。ただし、Mgr タスク数は 146\% 増加(56,625 → 139,376)した。この特徴は、\textbf{1 人あたりの効率が向上}する一方で、Mgr への負荷遷移が大きいことを示している。一方、\textbf{実験 B(分散化アプローチ)}では、SM 人数が 67\% 増加(238 → 398 名)し、SM 絶対負荷は 21\% 減少(94,420 → 74,369)した。ただし、SM 正規化負荷は 20\% 増加(2.67 → 3.21)し、Mgr タスク数は 47\% 増加(56,625 → 83,074)した。この特徴は、\textbf{総負荷を分散}できる一方で、1 人あたりの効率は低下することを示している。

\section{まとめ}

本章では、「中間管理職の負荷を構造的に軽減するにはどうすればよいか?」という問いに対して、2 つのアプローチを比較検証した。

第一に、\textbf{管理スパン削減は有効}であることを確認した。両実験とも、SM→Mgr スパンを削減することで SM の絶対負荷は減少した(実験 A: 14\% 減、実験 B: 21\% 減)。

第二に、\textbf{効率化アプローチ(実験 A)}では、SM の 1 タスクあたり負荷が 18\% 減少し、処理効率が向上した。ただし、Mgr のタスク数が 146\% 増加し、Mgr への負荷遷移が大きかった。

第三に、\textbf{分散化アプローチ(実験 B)}では、SM 人数が 67\% 増加し、絶対負荷は効率化より大きく減少した。ただし、1 タスクあたり負荷は 20\% 増加し、効率は低下した。

第四に、\textbf{トレードオフの存在}が明らかになった。効率化は「Mgr への負荷遷移」、分散化は「1 人あたり効率の低下」というトレードオフがある。

\clearpage

\begin{figure}[p]
\raggedright
\includegraphics[width=0.072\textwidth]{figures/max_load_transition.png}
\caption{実験 A:SimLoad の推移((a) 役職別最高値、(b) 正規化値)}
\label{fig:exp_a_transition}
\end{figure}

\begin{figure}[p]
\raggedright
\includegraphics[width=0.072\textwidth]{figures/dir_step_load_transition.png}
\caption{実験 B:SimLoad の推移((a) 役職別最高値、(b) 正規化値)}
\label{fig:exp_b_transition}
\end{figure}
